%% Preamble of the document starts here
\documentclass[12pt, a4paper]{article} %%doctype
\usepackage[top=1in, bottom=1in, left=1in, right=1in]{geometry} %%For formatting page
\usepackage{amsfonts, amsmath, amssymb, amsthm} %%For Math Symbols
\usepackage[none]{hyphenat}  %%Disable/Enable hypenation
%\usepackage{fancyvrb, fancyheadings}
\usepackage{fancyhdr}  %%For header and footer
\usepackage{graphicx} %% Inserting pics from pc
\graphicspath{C:\Users\user\Desktop\Proj_2023_LATeX 4th Sem}
\usepackage{float} %%Customizing the position of tables
\usepackage[nottoc, notlot, notlof]{tocbibind} %%tableofcontents
\usepackage{hyperref}  %% for referencing
\hypersetup{
    colorlinks=true,
    urlcolor=blue!50!black,
    pdftitle={How to write a math project}
}
\usepackage[utf8]{inputenc}
\usepackage[usenames,dvipsnames]{xcolor}
\pagecolor{white}
\usepackage{enumerate}
\usepackage{physics}
\usepackage{mathrsfs}  %for cursive writting
%\usepackage{romannum}
\usepackage{tikz, tcolorbox}
\usetikzlibrary{backgrounds}
\usepackage{pgfplots}
\pgfplotsset{compat=1.17}
\usepackage{halloweenmath}
\usepackage{lipsum}

%%block of code optional
% Define the inner product command
% \newcommand{\inner}[2]{\left\langle #1, #2 \right\rangle}

% % Define the theorem environment with a colored background
% \newtheoremstyle{colored}
%   {}{}{\itshape}{}{\bfseries}{.}{ }{\colorbox{blue!10}{\thmname{#1}\thmnumber{ #2}\thmnote{ (#3)}}}

% \theoremstyle{colored}
% \newtheorem{thm}{Theorem}[section]
%%block of code optional ends here





\pagestyle{fancy}    %%Styling the page as your wish
\fancyhead{}  %%Empty entity shows to remove the default header 
\fancyfoot{}   %%Empty entity shows to remove the default footer

%% Use this command to add header on left side.The optional argument [L] stands for Left
\fancyhead[L]{\href{ddugu.ac.in}{\includegraphics[scale=0.14]{DDU_Logo.png}}}

%% Use this command to add header on right side.The optional argument [R] stands for Right
\fancyhead[R]{\slshape Assignment:Measure Theory}

%%Use this command to add footer, [C] stands for page# shown in center
\fancyfoot[L]{\slshape By: \href{https://github.com/akhlak919}{Akhlak Ansari}}
\fancyfoot[R]{\bf \thepage}
%\fancyfoot[C]{\thepage}

%%Remove horizontal line in the header
%\renewcommand{\headrulewidth}{0pt}

%%Remove horizontal line in the footer
\renewcommand{\footrulewidth}{0.5pt}

%%Indentation related commands
\parindent0px  %% Make the paragraph indentation to zero
%\setlength{\parindent}{4em}
%\setlength{\parskip}{1em}
\renewcommand{\baselinestretch}{1.5}

\newcommand{\F}{\mathbb{F}}




%%Body of the document starts here
\begin{document}

\begin{titlepage}
    \begin{center}
        \vspace*{1cm}

        \Large{\bf Assignment}\\ 
        \Large{\textbf{Measure Theory}}
        

        \vspace*{1cm}

        \includegraphics[scale=1]{DDU_Logo.png}
        \vspace*{1cm}

        \begin{center}
            {\bf Department of Mathematics and Statistics\\
        DDU Gorakhpur University, Gorakhpur(India)}
        \end{center}
        
        \vfill %% automatic filling the all spces on the page

        {\tt \today}\\
    \end{center}
\end{titlepage}

\tableofcontents
\thispagestyle{empty}
\clearpage

\setcounter{page}{1}

% \vspace*{0.2cm}

\section{\slshape Assignment Questions}

\textcolor{red}{Instruction:} \textcolor{blue!85!yellow}{Attempt any four of the following questions -}

\begin{enumerate}
    \item Give examples of families of sets that are not an algebra or a $\sigma-\mbox{algebra}$.
    \item Prove that each closed subset of $\mathbb{R}^d$ is a $G_{\delta}$ and each open subset of $\mathbb{R}^d$ is $F_{\sigma} - $ set.
    \item Prove that Lebesgue outer measure on $\mathbb{R}^d$ is an outer measure, and it assigns to each $d-\mbox{dimensional}$ interval its volume.
    \item Let $(X, \mathscr{A})$ be a measurable space, and let $A$ be a subset of $X$ that belongs to $\mathscr{A}$. For a function $f:A \to \mathbb{R}$, the conditions 
    \begin{enumerate}[(a)]
        \item $f$ is measurable with respect to $\mathscr{A}$,
        \item for each open subset $U$ of $\mathbb{R}$ the set $f^{-1}(U)$ belongs to $\mathscr{A}$,
        \item for each closed subset $C$ of $\mathbb{R}$ the set $f^{-1}(C)$ belongs to $\mathscr{A}$, and
        \item for each Borel subset $B$ of $\mathbb{R}$ the set $f^{-1}(B)$ belongs to $\mathscr{A}$
    \end{enumerate}
    are equivalent.
    \item Let $(X, \mathscr{A}, \mu)$ be a measure space.Let $(Y, \mathscr{B})$ be a measurable space and let \\
    $f : (X, \mathscr{A}) \to (Y, \mathscr{B})$ be measurable.Let $g$ be an extended real valued $\mathscr{B}$ measurable function on $Y$ then $g$ is $\mu f^{-1}$ integrable if and only if $gof$ is $\mu-\mbox{integrable}$. If these are integrable then \[\int_Y gd(\mu f^{-1}) = \int_X gof du\]
\end{enumerate}

\pagebreak

\section{\slshape Solution of Assignment Questions}

\subsection{\slshape Solution of Question No. 1}

Let us define $\sigma\mbox{-algebra}$; 

A collection $\mathscr{A}$ of subsets of $X$ is a $\sigma\mbox{-algebra}$\footnote{The terms field and $\sigma\mbox{-field}$ are sometimes used in place of algebra and $\sigma\mbox{-algebra}$.} on $X$ If

\begin{enumerate}[a)]
    \item $X \in \mathscr{A}$,
    \item for each set $A$ that belongs to $\mathscr{A}$, the set $A^c$ belongs to $\mathscr{A}$.
    \item  for each infinite sequence $\left\{A_i\right\}$ of sets that belong to $\mathscr{A}$, the set $\displaystyle \bigcup_{i=1}^{\infty} A_i$ belongs to $\mathscr{A}$, and
    \item for each infinite sequence $\left\{A_i\right\}$ of sets that belong to $\mathscr{A}$, the set $\displaystyle \bigcap_{i=1}^{\infty} A_i$ belongs to $\mathscr{A}$
\end{enumerate}
Thus a $\sigma\mbox{-algebra}$ on $X$ is a family of subsets of $X$ that contains $X$ and is closed under complementation, under the formation of countable unions, and under the formation of countable intersections.


\subsubsection*{ Examples of Families of sets that are not $\sigma\mbox{-algebra}$ :}

\begin{enumerate}[1)]
    \item Let $X$ be an infinite set, and let $\mathscr{A}$ be the collection of all finite subsets of $X$.
    Then $\mathscr{A}$ does not contain $X$ and is not closed under complementation; hence it is not an algebra (or a $\sigma\mbox{-algebra}$) on $X$.
    \item  Let $X$ be an infinite set, and let $\mathscr{A}$ be the collection of all subsets $A$ of $X$ such that either $A$ or $A^c$ is finite. Then $\mathscr{A}$ is an algebra on $X$  but is not closed under the formation of countable unions; hence it is not a $\sigma\mbox{-algebra}$.
    \item Let $X$ be an uncountable set, and let $\mathscr{A}$ be the collection of all countable (i.e., finite or countably infinite) subsets of $X$. Then $\mathscr{A}$ does not contain $X$ and is not closed under complementation; hence it is not an algebra.
    \item Let $L$ be the collection of all finite disjoint unions of all intervals of the form:

    $(-\infty, a], (a, b], (b, \infty), \emptyset, \mathbb{R}$.
    
    Then $L$ is an algebra over $\mathbb{R}$, but not a $\sigma\mbox{-algebra}$ because union of sets $\displaystyle \left\{(0,\frac{i-1}{i}]\right\}$ for all $\displaystyle i \ge 1 = (0, 1) \notin L $.
\end{enumerate}

\subsection{\slshape Solution of Question No. 2}

Suppose that $F$ is a closed subset of $\mathbb{R}^d$. We need to construct a sequence
$\left\{U_n\right\}$ of open subsets of $\mathbb{R}^d$ such that $F = \cap_n U_n$. For this define $U_n$ by
\[U_n = \left\{x \in \mathbb{R}^d : \norm*{x-y} < \frac{1}{n}\hspace*{1cm} \mbox{for some $y$ in $F$}\right\}\]

(Note that $U_n$ is empty if $F$ is empty.) It is clear that each $U_n$ is open and that $F \subseteq \cap_n U_n$. The reverse inclusion follows from the fact that $F$ is closed (note that each point in $\cap_n U_n$ is the limit of a sequence of points in $F$). Hence each closed subset of $\mathbb{R}^d$ is a $G_\delta$.\\[2mm]
If $U$ is open, then $U^c$ is closed and so is a $G_\delta$. Thus there is a sequence $\left\{U_n\right\}$ of open sets such that $U^c = \cap_n U_n$. The sets $U^c_n$ are then closed, and $U = \cup_n U^c_n$; hence $U$ is an $F_\sigma$.


\subsection{\slshape Solution of Question No. 3}

We begin by verifying that $m^*$ is an outer measure. The relation $m^*(\emptyset) = 0$ holds, since for each positive number $\epsilon$ there is a sequence $\left\{(a_i,b_i)\right\}$ of open intervals (whose union necessarily includes $\emptyset$) such that $\sum_{i}^{}(b_i - a_i) < \epsilon$. For the monotonicity of $m^*$, note that if $A \subseteq B$, then each sequence of open intervals that covers $B$ also covers $A$, and so $m^*(A) \leq m^*(B)$. Now consider the countable subadditivity of $m^*$. Let $\left\{A_n\right\}_{n=1}^{\infty}$
be an arbitrary sequence of subsets of $\mathbb{R}$.
If $\sum_{n}^{}m^*(A_n)=+\infty$, then $m^*(\cup_{n} A_n) \leq \sum_{n}^{}m^*(A_n)$ certainly holds. 

So suppose that $\sum_{n}^{}m^*(A_n) < +\infty$, and let $\epsilon$ be an arbitrary positive number. For each $n$ choose a
sequence $\left\{(a_{n,i},b_{n,i})\right\}_{i=1}^{\infty}$ that covers $A_n$ and satisfies

\[\sum_{i=1}^{\infty}\left(b_{n,i}, a_{n,i}\right) < m^*(A_n) + \frac{\epsilon}{2^n}.\]

If we combine these sequences into one sequence $\left\{(a_j,b_j)\right\}$, then the combined sequence satisfies

\[
    \cup_n A_n \subseteq U_j(a_j, b_j)
\]
and
\[
    \sum_{j}^{}\left(b_j - a_j\right) < \sum_{n}^{}\left(m^*(A_n) + \frac{\epsilon}{2^n}\right) = \sum_{n}^{}m^*(A_n) + \epsilon
\]

These relations, together with the fact that $\epsilon$ is arbitrary, imply that $\displaystyle m^*(\cup_n A_n) \leq \sum_{n}^{}m^*(A_n)$. Thus $m^*$ is an outer measure.\\[1mm]

Now, Suppose that if $K$ is a compact $d\mbox{-dimensional}$
interval and if ${\left\{\mathbb{R}_i\right\}}_{i=1}^{\infty}$ is a sequence of bounded and open $d\mbox{-dimensional}$ intervals for which $\displaystyle K \subseteq \cup_{n=1}^{\infty} \mathbb{R}_i$, then there is a positive integer $n$ such that $\displaystyle K \subseteq \cup_{n=1}^{\infty} \mathbb{R}_i$, and $K$ can be decomposed into a finite collection $\left\{K_j\right\}$ of $d\mbox{-dimensional}$ intervals that overlap only on their boundaries and are such that for each $j$ the interior of $K_j$ is included in some $\mathbb{R}_i$ (where $i \leq n$). From this it follows that

\begin{align*}
    \boxed{\mbox{vol}(K) = \sum_{j}^{}\mbox{vol}(K_j) \leq \sum_{i}^{}\mbox{vol}(\mathbb{R}_i)}
\end{align*}

and hence that $\mbox{vol}(K) \leq m^*(K)$.

Overall, we shown that "Lebesgue outer measure on $\mathbb{R}^d$ is an outer measure, and it assigns to each $d\mbox{-dimensional}$ interval its volume".


\subsection{\slshape Solution of Question No. 4}
Let $\mathscr{F} = \left\{B \subseteq \mathbb{R} : f^{-1}(B) \in \mathscr{A} \right\}$. Then the fact that $f^{-1}(\mathbb{R}) = A$ and the identities
\[f^{-1}(B^c) = A - f^{-1}(B)\]
and
\[f^{-1}\left(\bigcup_{n} B_n\right) = \bigcup_{n} f^{-1}(B_n)\]

imply that $\mathscr{F}$ is a $\sigma\mbox{-algebra}$ on $\mathbb{R}$. To require that $f$ be measurable is to require that 
$\mathscr{F}$ contain all the intervals of the form $(-\infty,b]$ or equivalently (since $\mathscr{F}$ is a $\sigma\mbox{-algebra}$) to require that $\mathscr{F}$ include the $\sigma\mbox{-algebra}$ on $\mathbb{R}$ generated by these intervals.
Since the $\sigma\mbox{-algebra}$ generated by these intervals is the $\sigma\mbox{-algebra}$ of Borel subsets of $\mathbb{R}$ so it is obvious, conditions $(a)$ and $(d)$ are equivalent. However the $\sigma\mbox{-algebra}$ of Borel subsets of $\mathbb{R}$ is also generated by the collection of all open subsets of $\mathbb{R}$ and by the collection of all closed subsets of $\mathbb{R}$, and so conditions $(b)$ and $(c)$ are equivalent to the others.

\newpage

\section{References}

\begin{itemize}
    \item \slshape Measure Theory : Donald L. Cohn(Birkhauser edition)
    \item \slshape Measure, Integration \& Real Analysis : Sheldon Axler(Springer edition)
    \item \slshape JBH (https://math.stackexchange.com/users/91349/jbh), Example of an algebra which is not a $\sigma\mbox{-algebra}$., URL (version: 2013-08-22): https://math.stackexchange.com/q/473549
    \item \slshape For document related info visit \url{https://github.com/akhlak919/LaTeX_Stuffs}
\end{itemize}
\end{document}